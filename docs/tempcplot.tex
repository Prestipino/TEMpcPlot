%% Generated by Sphinx.
\def\sphinxdocclass{report}
\documentclass[letterpaper,10pt,english]{sphinxmanual}
\ifdefined\pdfpxdimen
   \let\sphinxpxdimen\pdfpxdimen\else\newdimen\sphinxpxdimen
\fi \sphinxpxdimen=.75bp\relax

\PassOptionsToPackage{warn}{textcomp}
\usepackage[utf8]{inputenc}
\ifdefined\DeclareUnicodeCharacter
% support both utf8 and utf8x syntaxes
  \ifdefined\DeclareUnicodeCharacterAsOptional
    \def\sphinxDUC#1{\DeclareUnicodeCharacter{"#1}}
  \else
    \let\sphinxDUC\DeclareUnicodeCharacter
  \fi
  \sphinxDUC{00A0}{\nobreakspace}
  \sphinxDUC{2500}{\sphinxunichar{2500}}
  \sphinxDUC{2502}{\sphinxunichar{2502}}
  \sphinxDUC{2514}{\sphinxunichar{2514}}
  \sphinxDUC{251C}{\sphinxunichar{251C}}
  \sphinxDUC{2572}{\textbackslash}
\fi
\usepackage{cmap}
\usepackage[T1]{fontenc}
\usepackage{amsmath,amssymb,amstext}
\usepackage{babel}



\usepackage{times}
\expandafter\ifx\csname T@LGR\endcsname\relax
\else
% LGR was declared as font encoding
  \substitutefont{LGR}{\rmdefault}{cmr}
  \substitutefont{LGR}{\sfdefault}{cmss}
  \substitutefont{LGR}{\ttdefault}{cmtt}
\fi
\expandafter\ifx\csname T@X2\endcsname\relax
  \expandafter\ifx\csname T@T2A\endcsname\relax
  \else
  % T2A was declared as font encoding
    \substitutefont{T2A}{\rmdefault}{cmr}
    \substitutefont{T2A}{\sfdefault}{cmss}
    \substitutefont{T2A}{\ttdefault}{cmtt}
  \fi
\else
% X2 was declared as font encoding
  \substitutefont{X2}{\rmdefault}{cmr}
  \substitutefont{X2}{\sfdefault}{cmss}
  \substitutefont{X2}{\ttdefault}{cmtt}
\fi


\usepackage[Bjarne]{fncychap}
\usepackage{sphinx}

\fvset{fontsize=\small}
\usepackage{geometry}


% Include hyperref last.
\usepackage{hyperref}
% Fix anchor placement for figures with captions.
\usepackage{hypcap}% it must be loaded after hyperref.
% Set up styles of URL: it should be placed after hyperref.
\urlstyle{same}
\addto\captionsenglish{\renewcommand{\contentsname}{Contents:}}

\usepackage{sphinxmessages}
\setcounter{tocdepth}{1}



\title{TEMpcPlot}
\date{Sep 18, 2020}
\release{alpha}
\author{C.\@{} Prestipino}
\newcommand{\sphinxlogo}{\vbox{}}
\renewcommand{\releasename}{Release}
\makeindex
\begin{document}

\pagestyle{empty}
\sphinxmaketitle
\pagestyle{plain}
\sphinxtableofcontents
\pagestyle{normal}
\phantomsection\label{\detokenize{index::doc}}

\index{module@\spxentry{module}!TEMpcPlot@\spxentry{TEMpcPlot}}\index{TEMpcPlot@\spxentry{TEMpcPlot}!module@\spxentry{module}}
The library TEMpcPlot has as object the treatments of a Sequence of
electropn diffraction cliches to obtain a three dimensional redcipriocal
lattice. The idea behind is to find a way of work for TEM, available on pc,
with graphycal approach (TEMpcPlot)
\begin{description}
\item[{The mode of use is relativelly simple :}] \leavevmode\begin{enumerate}
\sphinxsetlistlabels{\arabic}{enumi}{enumii}{}{.}%
\item {} 
create a SeqIm object

\end{enumerate}

\begin{sphinxVerbatim}[commandchars=\\\{\}]
\PYG{g+gp}{\PYGZgt{}\PYGZgt{}\PYGZgt{} }\PYG{n}{Ex1} \PYG{o}{=} \PYG{n}{TEMpcPlot}\PYG{p}{(}\PYG{n}{filelist}\PYG{p}{,} \PYG{n}{angles}\PYG{p}{)}
\end{sphinxVerbatim}
\begin{enumerate}
\sphinxsetlistlabels{\arabic}{enumi}{enumii}{}{.}%
\setcounter{enumi}{1}
\item {} 
construct a unindixed reciprocal lattice

\end{enumerate}

\begin{sphinxVerbatim}[commandchars=\\\{\}]
\PYG{g+gp}{\PYGZgt{}\PYGZgt{}\PYGZgt{} }\PYG{n}{Ex1}\PYG{o}{.}\PYG{n}{D3\PYGZus{}peaks}\PYG{p}{(}\PYG{n}{tollerance}\PYG{o}{=}\PYG{l+m+mi}{15}\PYG{p}{)}
\end{sphinxVerbatim}
\begin{enumerate}
\sphinxsetlistlabels{\arabic}{enumi}{enumii}{}{.}%
\setcounter{enumi}{2}
\item {} 
Index manually the reciprocal space

\end{enumerate}

\begin{sphinxVerbatim}[commandchars=\\\{\}]
\PYG{g+gp}{\PYGZgt{}\PYGZgt{}\PYGZgt{} }\PYG{n}{Ex1}\PYG{o}{.}\PYG{n}{EwP}\PYG{o}{.}\PYG{n}{plot}\PYG{p}{(}\PYG{p}{)}
\PYG{g+gp}{\PYGZgt{}\PYGZgt{}\PYGZgt{} }\PYG{n}{Ex1}\PYG{o}{.}\PYG{n}{EwP}\PYG{o}{.}\PYG{n}{graph}\PYG{o}{.}\PYG{n}{define\PYGZus{}axis}\PYG{p}{(}\PYG{l+s+s1}{\PYGZsq{}}\PYG{l+s+s1}{a}\PYG{l+s+s1}{\PYGZsq{}}\PYG{p}{,}\PYG{l+m+mi}{3} \PYG{p}{)}
\PYG{g+gp}{\PYGZgt{}\PYGZgt{}\PYGZgt{} }\PYG{n}{Ex1}\PYG{o}{.}\PYG{n}{EwP}\PYG{o}{.}\PYG{n}{graph}\PYG{o}{.}\PYG{n}{allign\PYGZus{}a}\PYG{p}{(}\PYG{p}{)}
\end{sphinxVerbatim}
\begin{enumerate}
\sphinxsetlistlabels{\arabic}{enumi}{enumii}{}{.}%
\setcounter{enumi}{3}
\item {} 
SET the cell for the EWALD peaks

\end{enumerate}

\begin{sphinxVerbatim}[commandchars=\\\{\}]
\PYG{g+gp}{\PYGZgt{}\PYGZgt{}\PYGZgt{} }\PYG{n}{Ex1}\PYG{o}{.}\PYG{n}{EwP}\PYG{o}{.}\PYG{n}{cr\PYGZus{}cond}\PYG{p}{(}\PYG{n}{operator}\PYG{o}{=}\PYG{l+s+s1}{\PYGZsq{}}\PYG{l+s+s1}{\PYGZsq{}}\PYG{p}{)}
\PYG{g+gp}{\PYGZgt{}\PYGZgt{}\PYGZgt{} }\PYG{n}{Ex1}\PYG{o}{.}\PYG{n}{EwP}\PYG{o}{.}\PYG{n}{set\PYGZus{}cell}\PYG{p}{(}\PYG{p}{)}
\PYG{g+gp}{\PYGZgt{}\PYGZgt{}\PYGZgt{} }\PYG{n}{Ex1}\PYG{o}{.}\PYG{n}{EwP}\PYG{o}{.}\PYG{n}{refine\PYGZus{}axes}\PYG{p}{(}\PYG{p}{)}
\end{sphinxVerbatim}
\begin{enumerate}
\sphinxsetlistlabels{\arabic}{enumi}{enumii}{}{.}%
\setcounter{enumi}{3}
\item {} 
reconstruct reciprocal space

\end{enumerate}

\begin{sphinxVerbatim}[commandchars=\\\{\}]
\PYG{g+gp}{\PYGZgt{}\PYGZgt{}\PYGZgt{} }\PYG{n}{Ex1}\PYG{o}{.}\PYG{n}{Ewp}\PYG{o}{.}\PYG{n}{create\PYGZus{}layer}\PYG{p}{(}\PYG{l+s+s1}{\PYGZsq{}}\PYG{l+s+s1}{k}\PYG{l+s+s1}{\PYGZsq{}}\PYG{p}{,} \PYG{l+m+mi}{1}\PYG{p}{)}
\end{sphinxVerbatim}

\end{description}


\bigskip\hrule\bigskip

\index{SeqIm (class in TEMpcPlot)@\spxentry{SeqIm}\spxextra{class in TEMpcPlot}}

\begin{fulllineitems}
\phantomsection\label{\detokenize{index:TEMpcPlot.SeqIm}}\pysiglinewithargsret{\sphinxbfcode{\sphinxupquote{class }}\sphinxcode{\sphinxupquote{TEMpcPlot.}}\sphinxbfcode{\sphinxupquote{SeqIm}}}{\emph{\DUrole{n}{filenames}}, \emph{\DUrole{n}{filesangle}\DUrole{o}{=}\DUrole{default_value}{None}}, \emph{\DUrole{o}{*}\DUrole{n}{args}}, \emph{\DUrole{o}{**}\DUrole{n}{kwords}}}{}
sequence of images

this class is supposed to use a sequence of image.
each element of the class is an image
\begin{quote}\begin{description}
\item[{Parameters}] \leavevmode\begin{itemize}
\item {} 
\sphinxstyleliteralstrong{\sphinxupquote{filenames}} (\sphinxstyleliteralemphasis{\sphinxupquote{list}}) \textendash{} \begin{enumerate}
\sphinxsetlistlabels{\roman}{enumi}{enumii}{}{)}%
\item {} 
list of string with filenames of image

\item {} 
string like “Ge*.dm3”

\item {} 
BIDS

\end{enumerate}


\item {} 
\sphinxstyleliteralstrong{\sphinxupquote{filesangle}} (\sphinxstyleliteralemphasis{\sphinxupquote{str}}) \textendash{} Human readable file with angles

\end{itemize}

\item[{Variables}] \leavevmode\begin{itemize}
\item {} 
\sphinxstyleliteralstrong{\sphinxupquote{EwP}} ({\hyperref[\detokenize{index:TEMpcPlot.EwaldPeaks}]{\sphinxcrossref{\sphinxstyleliteralemphasis{\sphinxupquote{TEMpcPlot.EwaldPeaks}}}}}) \textendash{} Ewald peaks 3D set of peaks

\item {} 
\sphinxstyleliteralstrong{\sphinxupquote{rot\_vect}} (\sphinxstyleliteralemphasis{\sphinxupquote{list}}) \textendash{} list of Rotation vector for each image

\item {} 
\sphinxstyleliteralstrong{\sphinxupquote{scale}} (\sphinxstyleliteralemphasis{\sphinxupquote{list}}) \textendash{} scale(magnification) of the images

\item {} 
\sphinxstyleliteralstrong{\sphinxupquote{ima}} (\sphinxstyleliteralemphasis{\sphinxupquote{TEMpcPlot.Mimage}}) \textendash{} current image of the sequence

\end{itemize}

\end{description}\end{quote}

\begin{sphinxadmonition}{note}{Note:}
\begin{DUlineblock}{0em}
\item[] Methods to use:
\item[] def find\_peaks(rad\_c=1.5, tr\_c=0.02, dist=None)
\item[] def D3\_peaks(stollerance)
\item[] def plot(log=False)
\end{DUlineblock}
\end{sphinxadmonition}
\index{D3\_peaks() (TEMpcPlot.SeqIm method)@\spxentry{D3\_peaks()}\spxextra{TEMpcPlot.SeqIm method}}

\begin{fulllineitems}
\phantomsection\label{\detokenize{index:TEMpcPlot.SeqIm.D3_peaks}}\pysiglinewithargsret{\sphinxbfcode{\sphinxupquote{D3\_peaks}}}{\emph{\DUrole{n}{tollerance}\DUrole{o}{=}\DUrole{default_value}{15}}}{}
sum and correct the peaks of all images
:param tollerance () = pixel tollerance to determine if a peak: in two images is the same peak.

\end{fulllineitems}

\index{find\_peaks() (TEMpcPlot.SeqIm method)@\spxentry{find\_peaks()}\spxextra{TEMpcPlot.SeqIm method}}

\begin{fulllineitems}
\phantomsection\label{\detokenize{index:TEMpcPlot.SeqIm.find_peaks}}\pysiglinewithargsret{\sphinxbfcode{\sphinxupquote{find\_peaks}}}{\emph{\DUrole{n}{rad\_c}\DUrole{o}{=}\DUrole{default_value}{1.5}}, \emph{\DUrole{n}{tr\_c}\DUrole{o}{=}\DUrole{default_value}{0.02}}, \emph{\DUrole{n}{dist}\DUrole{o}{=}\DUrole{default_value}{None}}}{}
findf the peak
allows to search again the peaks in all the image witht the same
parameter
\begin{quote}\begin{description}
\item[{Parameters}] \leavevmode\begin{itemize}
\item {} 
\sphinxstyleliteralstrong{\sphinxupquote{tr\_c}} (\sphinxstyleliteralemphasis{\sphinxupquote{float}}) \textendash{} total range coefficent the minimal intensity of the peak should be at list tr\_c*self.ima.max()

\item {} 
\sphinxstyleliteralstrong{\sphinxupquote{rad\_c}} (\sphinxstyleliteralemphasis{\sphinxupquote{float}}) \textendash{} coefficent in respect of the center radious peaks should be separate from at list self.rad*rad\_c

\item {} 
\sphinxstyleliteralstrong{\sphinxupquote{dist}} \textendash{} (float): maximum distance in pixel

\end{itemize}

\end{description}\end{quote}

Examples:
\textgreater{}\textgreater{}\textgreater{} Exp1.find\_peaks()

\end{fulllineitems}

\index{help() (TEMpcPlot.SeqIm method)@\spxentry{help()}\spxextra{TEMpcPlot.SeqIm method}}

\begin{fulllineitems}
\phantomsection\label{\detokenize{index:TEMpcPlot.SeqIm.help}}\pysiglinewithargsret{\sphinxbfcode{\sphinxupquote{help}}}{}{}
print class help

\end{fulllineitems}

\index{load() (TEMpcPlot.SeqIm class method)@\spxentry{load()}\spxextra{TEMpcPlot.SeqIm class method}}

\begin{fulllineitems}
\phantomsection\label{\detokenize{index:TEMpcPlot.SeqIm.load}}\pysiglinewithargsret{\sphinxbfcode{\sphinxupquote{classmethod }}\sphinxbfcode{\sphinxupquote{load}}}{\emph{\DUrole{n}{filename}}}{}
load a saved project
it is necessary that images remain in the same relative
position
:param filename: filename to open
:type filename: str
\subsubsection*{Examples}

\begin{sphinxVerbatim}[commandchars=\\\{\}]
\PYG{g+gp}{\PYGZgt{}\PYGZgt{}\PYGZgt{} }\PYG{n}{exp1} \PYG{o}{=} \PYG{n}{SeqIm}\PYG{o}{.}\PYG{n}{load}\PYG{p}{(}\PYG{n}{exp1}\PYG{o}{.}\PYG{n}{pkl}\PYG{p}{)}
\end{sphinxVerbatim}

\end{fulllineitems}

\index{plot() (TEMpcPlot.SeqIm method)@\spxentry{plot()}\spxextra{TEMpcPlot.SeqIm method}}

\begin{fulllineitems}
\phantomsection\label{\detokenize{index:TEMpcPlot.SeqIm.plot}}\pysiglinewithargsret{\sphinxbfcode{\sphinxupquote{plot}}}{\emph{\DUrole{n}{log}\DUrole{o}{=}\DUrole{default_value}{False}}, \emph{\DUrole{o}{*}\DUrole{n}{args}}, \emph{\DUrole{o}{**}\DUrole{n}{kwds}}}{}
plot the images of the sequences with peaks
\begin{quote}\begin{description}
\item[{Parameters}] \leavevmode\begin{itemize}
\item {} 
\sphinxstyleliteralstrong{\sphinxupquote{log}} (\sphinxstyleliteralemphasis{\sphinxupquote{Bool}}) \textendash{} plot logaritm of intyensity

\item {} 
\sphinxstyleliteralstrong{\sphinxupquote{anf keyworg directly of matplotlib plot}} (\sphinxstyleliteralemphasis{\sphinxupquote{aargs}}) \textendash{} 

\end{itemize}

\end{description}\end{quote}
\subsubsection*{Examples}

\begin{sphinxVerbatim}[commandchars=\\\{\}]
\PYG{g+gp}{\PYGZgt{}\PYGZgt{}\PYGZgt{} }\PYG{n}{Exp1}\PYG{o}{.}\PYG{n}{plot}\PYG{p}{(}\PYG{n}{log}\PYG{o}{=}\PYG{k+kc}{True}\PYG{p}{)}
\PYG{g+gp}{\PYGZgt{}\PYGZgt{}\PYGZgt{} }\PYG{n}{Exp1}\PYG{o}{.}\PYG{n}{plot}\PYG{p}{(}\PYG{k+kc}{True}\PYG{p}{)}
\PYG{g+gp}{\PYGZgt{}\PYGZgt{}\PYGZgt{} }\PYG{n}{Exp1}\PYG{o}{.}\PYG{n}{plot}\PYG{p}{(}\PYG{l+m+mi}{1}\PYG{p}{)}
\PYG{g+gp}{\PYGZgt{}\PYGZgt{}\PYGZgt{} }\PYG{n}{Exp1}\PYG{o}{.}\PYG{n}{plot}\PYG{p}{(}\PYG{l+m+mi}{0}\PYG{p}{)}
\PYG{g+gp}{\PYGZgt{}\PYGZgt{}\PYGZgt{} }\PYG{n}{Exp1}\PYG{o}{.}\PYG{n}{plot}\PYG{p}{(}\PYG{n}{vmin} \PYG{o}{=} \PYG{l+m+mi}{10}\PYG{p}{,} \PYG{p}{)}
\end{sphinxVerbatim}

\end{fulllineitems}

\index{save() (TEMpcPlot.SeqIm method)@\spxentry{save()}\spxextra{TEMpcPlot.SeqIm method}}

\begin{fulllineitems}
\phantomsection\label{\detokenize{index:TEMpcPlot.SeqIm.save}}\pysiglinewithargsret{\sphinxbfcode{\sphinxupquote{save}}}{\emph{\DUrole{n}{filesave}}}{}~\begin{quote}

save the project to open later
formats available: None: pickel format good for python
\end{quote}
\begin{quote}\begin{description}
\item[{Parameters}] \leavevmode
\sphinxstyleliteralstrong{\sphinxupquote{filename}} (\sphinxstyleliteralemphasis{\sphinxupquote{str}}) \textendash{} filename to save

\end{description}\end{quote}
\subsubsection*{Examples}

\begin{sphinxVerbatim}[commandchars=\\\{\}]
\PYG{g+gp}{\PYGZgt{}\PYGZgt{}\PYGZgt{} }\PYG{n}{Exp1}\PYG{o}{.}\PYG{n}{save}\PYG{p}{(}\PYG{l+s+s1}{\PYGZsq{}}\PYG{l+s+s1}{exp1}\PYG{l+s+s1}{\PYGZsq{}}\PYG{p}{)}
\end{sphinxVerbatim}

\end{fulllineitems}


\end{fulllineitems}

\index{EwaldPeaks (class in TEMpcPlot)@\spxentry{EwaldPeaks}\spxextra{class in TEMpcPlot}}

\begin{fulllineitems}
\phantomsection\label{\detokenize{index:TEMpcPlot.EwaldPeaks}}\pysiglinewithargsret{\sphinxbfcode{\sphinxupquote{class }}\sphinxcode{\sphinxupquote{TEMpcPlot.}}\sphinxbfcode{\sphinxupquote{EwaldPeaks}}}{\emph{\DUrole{n}{positions}}, \emph{\DUrole{n}{intensity}}, \emph{\DUrole{n}{rot\_vect}\DUrole{o}{=}\DUrole{default_value}{None}}, \emph{\DUrole{n}{axes}\DUrole{o}{=}\DUrole{default_value}{None}}, \emph{\DUrole{n}{set\_cell}\DUrole{o}{=}\DUrole{default_value}{True}}}{}
Set of peaks position and intensity

this class manages peaks position and intensity and the methods related to
lattice indexing and refinement
could be created as an attribute EwP of a SeqIm class by using methods D3\_peaks
or by sum with an another EwaldPeaks class with the same first image
\subsubsection*{Example}

\textgreater{}\textgreater{}\textgreater{}Exp1.D3\_peaks(tollerance=5)
\textgreater{}\textgreater{}\textgreater{}EWT= Exp1.EwP +  Exp2.EwP
\begin{quote}\begin{description}
\item[{Parameters}] \leavevmode\begin{itemize}
\item {} 
\sphinxstyleliteralstrong{\sphinxupquote{positions}} (\sphinxstyleliteralemphasis{\sphinxupquote{list}}) \textendash{} list containing the coordonates of peaks

\item {} 
\sphinxstyleliteralstrong{\sphinxupquote{intensity}} (\sphinxstyleliteralemphasis{\sphinxupquote{list}}) \textendash{} list containing the intensity of peaks

\end{itemize}

\item[{Variables}] \leavevmode\begin{itemize}
\item {} 
\sphinxstyleliteralstrong{\sphinxupquote{pos}} (\sphinxstyleliteralemphasis{\sphinxupquote{list}}) \textendash{} Ewald peaks 3D set of peaks

\item {} 
\sphinxstyleliteralstrong{\sphinxupquote{int}} (\sphinxstyleliteralemphasis{\sphinxupquote{list}}) \textendash{} list of Rotation vector for each image

\item {} 
\sphinxstyleliteralstrong{\sphinxupquote{pos\_cal}} (\sphinxstyleliteralemphasis{\sphinxupquote{np.array}}) \textendash{} array witht he position in the new basis

\item {} 
\sphinxstyleliteralstrong{\sphinxupquote{rMT}} (\sphinxstyleliteralemphasis{\sphinxupquote{np.array}}) \textendash{} reciprocal metric tensor

\item {} 
\sphinxstyleliteralstrong{\sphinxupquote{cell}} (\sphinxstyleliteralemphasis{\sphinxupquote{dict}}) \textendash{} a dictionary witht the value of
real space cell

\end{itemize}

\end{description}\end{quote}
\index{cr\_cond() (TEMpcPlot.EwaldPeaks method)@\spxentry{cr\_cond()}\spxextra{TEMpcPlot.EwaldPeaks method}}

\begin{fulllineitems}
\phantomsection\label{\detokenize{index:TEMpcPlot.EwaldPeaks.cr_cond}}\pysiglinewithargsret{\sphinxbfcode{\sphinxupquote{cr\_cond}}}{\emph{\DUrole{n}{operator}\DUrole{o}{=}\DUrole{default_value}{None}}, \emph{\DUrole{n}{lim}\DUrole{o}{=}\DUrole{default_value}{None}}}{}
define filtering condition

fuch function create a function that filter the data following the condition

\end{fulllineitems}

\index{create\_layer() (TEMpcPlot.EwaldPeaks method)@\spxentry{create\_layer()}\spxextra{TEMpcPlot.EwaldPeaks method}}

\begin{fulllineitems}
\phantomsection\label{\detokenize{index:TEMpcPlot.EwaldPeaks.create_layer}}\pysiglinewithargsret{\sphinxbfcode{\sphinxupquote{create\_layer}}}{\emph{\DUrole{n}{hkl}}, \emph{\DUrole{n}{n}}, \emph{\DUrole{n}{size}\DUrole{o}{=}\DUrole{default_value}{0.25}}, \emph{\DUrole{n}{toll}\DUrole{o}{=}\DUrole{default_value}{0.15}}, \emph{\DUrole{n}{mir}\DUrole{o}{=}\DUrole{default_value}{0}}, \emph{\DUrole{n}{spg}\DUrole{o}{=}\DUrole{default_value}{None}}}{}
create a specific layer
create a reciprocal space layer
\begin{quote}\begin{description}
\item[{Parameters}] \leavevmode\begin{itemize}
\item {} 
\sphinxstyleliteralstrong{\sphinxupquote{hkl}} (\sphinxstyleliteralemphasis{\sphinxupquote{str}}) \textendash{} constant index for the hkl plane to plot, format(‘k’)

\item {} 
\sphinxstyleliteralstrong{\sphinxupquote{n}} (\sphinxstyleliteralemphasis{\sphinxupquote{float}}\sphinxstyleliteralemphasis{\sphinxupquote{, }}\sphinxstyleliteralemphasis{\sphinxupquote{int}}) \textendash{} value of hkl

\item {} 
\sphinxstyleliteralstrong{\sphinxupquote{size}} (\sphinxstyleliteralemphasis{\sphinxupquote{float}}) \textendash{} intensity scaling \begin{description}
     \item[-{ if positive, scale intensity of each peaks respect the max}]
      \item[-{ if negative, scale a common value for all peaks}]


\end{description}


\item {} 
\sphinxstyleliteralstrong{\sphinxupquote{tollerance}} (\sphinxstyleliteralemphasis{\sphinxupquote{float}}) \textendash{} exclude from the plot peaks at higher distance

\item {} 
\sphinxstyleliteralstrong{\sphinxupquote{spg}} (\sphinxstyleliteralemphasis{\sphinxupquote{str}}) \textendash{} allows to index the peaks, and check if they are extinted

\end{itemize}

\end{description}\end{quote}

\end{fulllineitems}

\index{plot() (TEMpcPlot.EwaldPeaks method)@\spxentry{plot()}\spxextra{TEMpcPlot.EwaldPeaks method}}

\begin{fulllineitems}
\phantomsection\label{\detokenize{index:TEMpcPlot.EwaldPeaks.plot}}\pysiglinewithargsret{\sphinxbfcode{\sphinxupquote{plot}}}{}{}
open a D3plot graph

\end{fulllineitems}

\index{plot\_reduce() (TEMpcPlot.EwaldPeaks method)@\spxentry{plot\_reduce()}\spxextra{TEMpcPlot.EwaldPeaks method}}

\begin{fulllineitems}
\phantomsection\label{\detokenize{index:TEMpcPlot.EwaldPeaks.plot_reduce}}\pysiglinewithargsret{\sphinxbfcode{\sphinxupquote{plot\_reduce}}}{\emph{\DUrole{n}{tollerance}\DUrole{o}{=}\DUrole{default_value}{0.1}}, \emph{\DUrole{n}{condition}\DUrole{o}{=}\DUrole{default_value}{None}}}{}
plot collapsed reciprocal space

\end{fulllineitems}

\index{refine\_angles() (TEMpcPlot.EwaldPeaks method)@\spxentry{refine\_angles()}\spxextra{TEMpcPlot.EwaldPeaks method}}

\begin{fulllineitems}
\phantomsection\label{\detokenize{index:TEMpcPlot.EwaldPeaks.refine_angles}}\pysiglinewithargsret{\sphinxbfcode{\sphinxupquote{refine\_angles}}}{\emph{\DUrole{n}{axes}\DUrole{o}{=}\DUrole{default_value}{None}}, \emph{\DUrole{n}{tollerance}\DUrole{o}{=}\DUrole{default_value}{0.1}}, \emph{\DUrole{n}{zero\_tol}\DUrole{o}{=}\DUrole{default_value}{0.1}}}{}
refine reciprocal cell basis
refine the reciprocal cell basis in respect to data that are
indexed in the tollerance range.

\end{fulllineitems}

\index{refine\_axang() (TEMpcPlot.EwaldPeaks method)@\spxentry{refine\_axang()}\spxextra{TEMpcPlot.EwaldPeaks method}}

\begin{fulllineitems}
\phantomsection\label{\detokenize{index:TEMpcPlot.EwaldPeaks.refine_axang}}\pysiglinewithargsret{\sphinxbfcode{\sphinxupquote{refine\_axang}}}{\emph{\DUrole{n}{axes}\DUrole{o}{=}\DUrole{default_value}{None}}, \emph{\DUrole{n}{tollerance}\DUrole{o}{=}\DUrole{default_value}{0.1}}, \emph{\DUrole{n}{zero\_tol}\DUrole{o}{=}\DUrole{default_value}{0.1}}}{}
refine reciprocal cell basis
refine the reciprocal cell basis in respect to data that are
indexed in the tollerance range.

\end{fulllineitems}

\index{refine\_axes() (TEMpcPlot.EwaldPeaks method)@\spxentry{refine\_axes()}\spxextra{TEMpcPlot.EwaldPeaks method}}

\begin{fulllineitems}
\phantomsection\label{\detokenize{index:TEMpcPlot.EwaldPeaks.refine_axes}}\pysiglinewithargsret{\sphinxbfcode{\sphinxupquote{refine\_axes}}}{\emph{\DUrole{n}{axes}\DUrole{o}{=}\DUrole{default_value}{None}}, \emph{\DUrole{n}{tollerance}\DUrole{o}{=}\DUrole{default_value}{0.1}}}{}
refine reciprocal cell basis
refine the reciprocal cell basis in respect to data that are
indexed in the tollerance range.

\end{fulllineitems}

\index{save() (TEMpcPlot.EwaldPeaks method)@\spxentry{save()}\spxextra{TEMpcPlot.EwaldPeaks method}}

\begin{fulllineitems}
\phantomsection\label{\detokenize{index:TEMpcPlot.EwaldPeaks.save}}\pysiglinewithargsret{\sphinxbfcode{\sphinxupquote{save}}}{\emph{\DUrole{n}{filename}}, \emph{\DUrole{n}{dictionary}\DUrole{o}{=}\DUrole{default_value}{False}}}{}
save EwP
formats available:
\begin{quote}

None: pickel format good for python
Idx : for Ind\_x
\end{quote}

\end{fulllineitems}

\index{set\_cell() (TEMpcPlot.EwaldPeaks method)@\spxentry{set\_cell()}\spxextra{TEMpcPlot.EwaldPeaks method}}

\begin{fulllineitems}
\phantomsection\label{\detokenize{index:TEMpcPlot.EwaldPeaks.set_cell}}\pysiglinewithargsret{\sphinxbfcode{\sphinxupquote{set\_cell}}}{\emph{\DUrole{n}{axes}\DUrole{o}{=}\DUrole{default_value}{None}}, \emph{\DUrole{n}{axes\_std}\DUrole{o}{=}\DUrole{default_value}{None}}, \emph{\DUrole{n}{tollerance}\DUrole{o}{=}\DUrole{default_value}{0.1}}, \emph{\DUrole{n}{cond}\DUrole{o}{=}\DUrole{default_value}{None}}}{}
calculation of the cell
effect the calculation to obtain the cell
:param axis:
\begin{quote}
\begin{description}
\item[{the new reciprocal basis to be used in the format}] \leavevmode
\begin{DUlineblock}{0em}
\item[] np.array{[}{[}a1, b1, c1{]},
\item[]
\begin{DUlineblock}{\DUlineblockindent}
\item[] {[}a2, b2, c2{]},
\item[] {[}a3, b3, c3{]}{]}
\end{DUlineblock}
\end{DUlineblock}

\end{description}

if axis is not inoput the programm seach if a new basis
has been defined graphically
\end{quote}
\begin{quote}\begin{description}
\item[{Returns}] \leavevmode
nothing

\item[{Variables}] \leavevmode\begin{itemize}
\item {} 
\sphinxstyleliteralstrong{\sphinxupquote{self.rMT}} (\sphinxstyleliteralemphasis{\sphinxupquote{np.array}}) \textendash{} reciprocal metric tensor

\item {} 
\sphinxstyleliteralstrong{\sphinxupquote{self.cell}} (\sphinxstyleliteralemphasis{\sphinxupquote{dict}}) \textendash{} a dictionary witht the value of
real space cell

\end{itemize}

\end{description}\end{quote}
\begin{description}
\item[{“””}] \leavevmode\begin{quote}

self.rMT    : reciprocal metric tensor
\end{quote}
\begin{description}
\item[{self.cell}] \leavevmode{[}a dictionary witht the value of{]}
real space cell

\end{description}

\end{description}

\end{fulllineitems}


\end{fulllineitems}

\index{SpaceGroup() (in module TEMpcPlot)@\spxentry{SpaceGroup()}\spxextra{in module TEMpcPlot}}

\begin{fulllineitems}
\phantomsection\label{\detokenize{index:TEMpcPlot.SpaceGroup}}\pysiglinewithargsret{\sphinxcode{\sphinxupquote{TEMpcPlot.}}\sphinxbfcode{\sphinxupquote{SpaceGroup}}}{\emph{\DUrole{n}{SGSymbol}}}{}
Print the output of SpcGroup in a nicely formatted way.
\begin{quote}\begin{description}
\item[{Parameters}] \leavevmode
\sphinxstyleliteralstrong{\sphinxupquote{SGSymbol}} \textendash{} space group symbol (string) with spaces between axial fields

\item[{Returns}] \leavevmode
nothing

\end{description}\end{quote}

\end{fulllineitems}

\index{SytSym() (in module TEMpcPlot)@\spxentry{SytSym()}\spxextra{in module TEMpcPlot}}

\begin{fulllineitems}
\phantomsection\label{\detokenize{index:TEMpcPlot.SytSym}}\pysiglinewithargsret{\sphinxcode{\sphinxupquote{TEMpcPlot.}}\sphinxbfcode{\sphinxupquote{SytSym}}}{\emph{\DUrole{n}{XYZ}}, \emph{\DUrole{n}{SGData}}}{}
Generates the number of equivalent positions and a site symmetry code for a specified coordinate and space group
\begin{quote}\begin{description}
\item[{Parameters}] \leavevmode\begin{itemize}
\item {} 
\sphinxstyleliteralstrong{\sphinxupquote{XYZ}} \textendash{} an array, tuple or list containing 3 elements: x, y \& z

\item {} 
\sphinxstyleliteralstrong{\sphinxupquote{SGData}} \textendash{} from SpcGroup

\end{itemize}

\item[{Returns}] \leavevmode
a four element tuple:
\begin{itemize}
\item {} 
The 1st element is a code for the site symmetry (see GetKNsym)

\item {} 
The 2nd element is the site multiplicity

\item {} 
Ndup number of overlapping operators

\item {} 
dupDir Dict \sphinxhyphen{} dictionary of overlapping operators

\end{itemize}

\end{description}\end{quote}

\end{fulllineitems}

\index{pt\_p() (in module TEMpcPlot)@\spxentry{pt\_p()}\spxextra{in module TEMpcPlot}}

\begin{fulllineitems}
\phantomsection\label{\detokenize{index:TEMpcPlot.pt_p}}\pysiglinewithargsret{\sphinxcode{\sphinxupquote{TEMpcPlot.}}\sphinxbfcode{\sphinxupquote{pt\_p}}}{\emph{\DUrole{n}{atom}}, \emph{\DUrole{n}{property}}}{}
Atomic properties
:param property: property type
:type property: str
\begin{quote}\begin{description}
\item[{Returns}] \leavevmode
property of the atoms

\item[{Return type}] \leavevmode
floats, string

\end{description}\end{quote}
\subsubsection*{Notes}

‘At\_w’    : atomic weight
‘Z’       : atomic number
‘cov\_r’   : covalent radii
‘sym’     : atomic symbol
‘e\_conf’  : electronic conf.
‘ox\_st’   : oxydation state
‘bon\_dis’ : typical bond distances
‘edges’   : x\sphinxhyphen{}ray edges
————————\sphinxhyphen{}

Examples:
\textgreater{}\textgreater{}\textgreater{}pt\_p(34, ‘sym’)

\textgreater{}\textgreater{}\textgreater{}pt\_p(‘Cu’, ‘At\_w’)

\end{fulllineitems}



\chapter{Indices and tables}
\label{\detokenize{index:indices-and-tables}}\begin{itemize}
\item {} 
\DUrole{xref,std,std-ref}{genindex}

\item {} 
\DUrole{xref,std,std-ref}{modindex}

\item {} 
\DUrole{xref,std,std-ref}{search}

\end{itemize}


\renewcommand{\indexname}{Python Module Index}
\begin{sphinxtheindex}
\let\bigletter\sphinxstyleindexlettergroup
\bigletter{t}
\item\relax\sphinxstyleindexentry{TEMpcPlot}\sphinxstyleindexpageref{index:\detokenize{module-TEMpcPlot}}
\end{sphinxtheindex}

\renewcommand{\indexname}{Index}
\printindex
\end{document}